
%%%%%%%%%%%%%%%%%%%%% Discussion %%%%%%%%%%%%%%%%%%%%%%%%%%%%%%%%%

\section{Discussion}

Gene expression profiling has yet provided little insight into molecular mechanism underlying genomic variation in CLL. The main hurdle is the high phenotypic and genotypic heterogeneity. To overcome these challenges we collected and analysed 184 CLL transcriptomes, representing the largest CLL transcriptome dataset so far. Indeed we were able to identify gene expression signatures for the most prevalent genetic variants and could establish the hypermutation status of IGHV and trisomy12 as main driver of gene expression variability. In contrast to previous findings we found 9.5 \% of gene expression variance associated with the IGHV status, revealing a higher impact on transcriptional changes than previously assumed \citep{Ferreira2014a}. We find clear associations between genotype data and gene expression and show that these can be used to investigate underlying mechanisms. So far global expression pattern in CLL have not been related to major genetic variation \citep{Ferreira2014a}. We could further validate our findings by the expression of some marker genes as CD38 and other previously reported associations. Using unsupervised clustering on the most variable genes, we could also distinguish samples by their associations to methylation groups, which support for the importance and function of this refined sample classification.  Altogether we show how this dataset can provide an important resource to complement current understanding of CLLs by supporting previous findings as well as providing novel hypothesis on molecular mechanisms underlying the pathogenesis of CLL.\\
\\
As CLL is often characterized by multiple co-occurring genetic lesions we further investigated the effect of genetic interaction on expression changes. Thereby we focused on the main driver of gene expression variability Trisomy12 and the IGHV status. We could not only identify numerous genes with specific expression pattern due to this interaction, but were also able to cluster them by different types of epistatic interactions like buffering, suppression and inversion. This concept of mixed epistasis between genetic lesions reflected on gene expression has been described in yeast before \citep{Sameith2015}. Here we show the functional role and mechanistic importance of interaction between variants. We apply the concept of mixed epistasis and show how this can be used to unravel another layer of complexity in the diversity of tumorigenesis. This systematic approach was used to identify potential pathways as G2M checkpoint pathway, that collaboratively connect the IGHV status and trisomy12 and can help us to understand subtype specific differences. We further find epistatic interaction pattern in ex-vivo drug response data. Sample with hypermutated IGHV and trisomy12 are less sensitive toward a group of drugs indirectly targeting DNA. Our findings from expression analysis suggest more functional DNA damage machinery in these sample and thus provide a mechanistic explanation for this phenotype.   
  

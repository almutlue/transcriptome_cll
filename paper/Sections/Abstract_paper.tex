%%%%%%%%%%%%%%%%%%%%%%% Abstract %%%%%%%%%%%%%%%%%%%%%%%%%%%%%%%%%%%%%%%%%%%%

\section{Abstract}

While recurrent (driver) mutations in CLL have been extensively catalogued, how they affect the disease phenotypes individually or collaboratively remains poorly characterized. To address this need, we performed RNA sequencing on 184 CLL patients and linked gene expression changes to genomic variations. \\

We identified robust and previously unknown gene expression signatures for recurrent copy number variations (including trisomy12, del11q22.3, del17p13, del18p12 and gain8q24), gene mutations (TP53, BRAF and SF3B1) and the somatic hypermutation status of the immunoglobulin heavy-chain variable region (IGHV). The most profound gene expression changes are associated with IGHV (3275 genes), methylation groups and trisomy 12 (3557 genes), which were apparent on unsupervised clustering. Gene set enrichment… (ighv, Tri12). We also demonstrate genetic interactions reflected in the transcriptional profiles. IGHV showed a mixed epistasis interaction with trisomy 12, suggesting…. Different groups of genes showed coordinated changes representing buffering, suppression and inversion of their expression phenotypes. Mixed epistasis involving IGHV and trisomy was also supported by drug response phenotypes and global methylation profiles. \\

In summary, our study provides a comprehensive reference data set for gene expression in CLL. We show that IGHV mutation status, recurrent gene mutations and CNVs drive gene expression and exhibit specific expression pattern. This includes epistatic interaction between trisomy12 and IGHV. Using a novel way to describe coordinated changes we can group genes into sets related to buffering, inversion and suppression.
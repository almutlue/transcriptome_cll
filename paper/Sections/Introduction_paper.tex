
%%%%%%%%%%%%%%%%%%%%% Introduction %%%%%%%%%%%%%%%%%%%%%%%%%%%%%%%%%

\section{Introduction}

Chronic lymphocytic leukemia (CLL) is a heterogeneous disease with variable course and tumour dynamics \citep{Zenz2010, Fabbri2016}. Large cohort studies using next generation sequencing technologies uncovered a highly variable genotypic and epigenetic landscape underlying this heterogeneity. The mutation status of IGHV and TP53 mutations have been established as major prognostic factors. The functional role of most genomic variants remain incompletely characterized, but the incidence of many gene mutations suggest a strong selective advantage. The majority of CLL patients have more than one recurrent mutation or copy number variation. Numerous characteristic patterns of co-occurrence have been identified between different genomic variations e.g. between the IGHV mutation status, Notch1, SF3B1 and ATM mutation. Nevertheless the functional impact and the interplay which determine phenotypic changes are still incompletely understood. \\

Transcriptome profiling techniques, such as RNA-sequencing, capture a snapshot of the dynamic RNA expression on a genome-wide scale, reflecting the diversity of cellular states and underlying regulatory mechanisms. Previous transcriptomic studies on CLL samples showed variability in gene expression, but suggested that major genetic subgroups may not be discernible by analyzing gene expression \citep{Ferreira2014a, Rosenwald2001}. \\
 
A recent comparison between CLL and normal B cell samples showed large expression changes that were associated with metabolic pathways, ribosome and spliceosome \citep{Ferreira2014a}.  While CLL showed high gene expression variability, few robust gene expression changes were associated with genotypes. Associations were found for the IGHV status, which accounted for $1.5\%$ of the overall variance. The analysis revealed two subgroups termed c1/c2, which were independent of known molecular disease groups. These subgroups were independent of the genomic background, but showed clear expression changes, which were associated with clinical outcome \citep{Ferreira2014a}.
Later expression changes were linked to potential technical effects of sample processing. In a large pan cancer analysis Dvinge et al. found that blood collection procedures may rapidly change the transcriptional and posttranscriptional landscapes of hematopoietic cells, resulting in biased activation of specific biological pathways \citep{Dvinge2014}. \\


Variable RNA expression of CLL patients could be a consequence of the cell of origin and e.g. heterogeneous genetic landscape including the impact of driver mutations on transcriptional programmes and thus provide the opportunity to dissect the functional roles through transcriptomic profiling. However, associations between recurrent genomic variations and transcriptomic changes have been surprisingly rare considering the targeting of key processes including BCR signalling (IGHV), DNA damage pathways (p53, ATM) and NOTCH signalling (NOTCH1, FBXW7, MED12) \citep{Landau2013, Puente2011a, Rossi2016a}. Even for major biological disease subgroups as the hypermutation status of IGHV an initial study could only find differences in the overall variance of gene expression between U-CLL and M-CLL, but no clear expression profiles \citep{Ecker2015}. \\

To understand the biology of CLL and study the impact of genetic variants on gene expression in CLL, we profiled 184 CLL samples using RNA-sequencing and searched for molecular features underlying gene expression variability in CLL. We aimed to identify the expression signatures for the most prevalent gene mutations and copy number changes, which includes the IGHV hypermutation status, trisomy12, gene mutations in TP53, ATM, BRAF, SF3B1, Notch1 and MED12 and CNVs as del17p13, del11q22.3, del8p12 and gain8q24. Besides transcriptional profiling of these single variants we looked for pattern of genomic interaction within gene expression. 



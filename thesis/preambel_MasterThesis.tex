

\usepackage{graphicx}
\usepackage[usenames]{color}

\usepackage[english]{babel}  
\usepackage[automark]{scrlayer-scrpage}
\usepackage{setspace}

\usepackage{float}
\usepackage{subcaption}  %%% newer version that can be used instead of subfigure
\usepackage[section]{placeins}

\usepackage{geometry}
\usepackage{url}


\usepackage{texshade} % MSA prints
\usepackage[]{amsmath} % Amsmath - Mathematik Basispaket
\usepackage{amssymb}

%%%%%%%%%% bibliography %%%%%%%%%%%%%%%%%%%%%%%%%%%%%%%
\usepackage[authoryear, sort, square]{natbib}
\bibliographystyle{cell}
%\usepackage[%
%	square,	% for square brackets;
%	comma,	% to use commas as separaters;
%	numbers,	% for numerical citations;
%	sort,		% orders multiple citations into the sequence in which they appear in the list of references;
%	sort&compress,    % as sort but in addition multiple numerical citations
%]{natbib}

%spaces between single entries
%\setlength{\bibsep}{3pt}
%%%%%%%%%%%%%%%%%%%%%%%%%%%%%%%%%%%%%%%%%%%%%%%%%%%%%


%\usepackage{subfig} % Layout wird weiter unten festgelegt !
%\usepackage{subfigure} % Layout wird weiter unten festgelegt !
% Aussehen der Captions fuer subfigures (subfig-Paket)
\usepackage[english,intoc]{nomencl}
%\usepackage{url} % Setzen von URLs. In Verbindung mit hyperref sind diese auch aktive Links.
\usepackage{ae} %erkennt umlaute
% sz
%\usepackage[latin1]{inputenc} 
\usepackage[utf8]{inputenc}

%\usepackage{caption}
\usepackage[font={sf,bf,small},textfont=md,textfont=md]{caption} 

% Aussehen der Captions
\captionsetup{
   margin = 5pt,
   font={sf,bf,small},
   textfont = {md,small},
   format = default, %singlespace% oder 'hang'
   indention = 0em,  % Einruecken der Beschriftung
   %labelsep = colon, %period, space, quad, newline
   %justification = RaggedRight, % justified, centering
   singlelinecheck = true, % false (true=bei einer Zeile immer zentrieren)
   position = bottom %top
}
%%% Bugfix Workaround
\DeclareCaptionOption{parskip}[]{}
\DeclareCaptionOption{parindent}[]{}
%\DeclareCaptionLabelFormat{andtable}{#1~#2 \& \tablename~\thetable}

% Aussehen der Captions fuer subfigures (subfig-Paket)
%\captionsetup[subfloat]{%
%   margin = 10pt,
%   font = {small,rm},
%   labelfont = {small,bf},
%   format = default, % oder 'hang'
%   indention = 0em,  % Einruecken der Beschriftung
%   labelsep = space, %period, space, quad, newline
%   justification = RaggedRight, % justified, centering
%   singlelinecheck = true, % false (true=bei einer Zeile immer zentrieren)
%   position = bottom, %top
%   labelformat = parens % simple, empty % Wie die Bezeichnung gesetzt wird
% }
%\setlength{\tabcolsep}{4pt} %distance between columns
%\renewcommand{\arraystretch}{0.75} %distance between rows in tables

\usepackage{setspace}


%\renewcommand{\baselinestretch}{1.5}%line spacing
%more figures allowed per page
\renewcommand\floatpagefraction{1}
\renewcommand\topfraction{1}
\renewcommand\bottomfraction{1}
\renewcommand\textfraction{0}   
\setcounter{totalnumber}{50}
\setcounter{topnumber}{50}
\setcounter{bottomnumber}{50}

%index
\usepackage{makeidx}
\makeindex


\newcommand{\nom}[2]{#1 \ \dotfill \ #2 \\}

%\usepackage{fancyhdr}
%\pagestyle{fancy}

%%%%%%%%%%%%%%%%% Page layout %%%%%%%%%%%%%%%%%%%%%%%%%%%

%\addtolength{\textheight}{3cm}
%\voffset-0.5cm
%\hoffset-1cm
%\addtolength{\textwidth}{2cm}


\geometry{a4paper, top=25mm, left=25mm, right=25mm, bottom=25mm,
	headsep=10mm, footskip=12mm}


%%%%%%%%%%%%%%%% Figure Path %%%%%%%%%%%%%%%%%%%%%%

%\usepackage{standalone}
%\standaloneconfig{mode=buildnew}
\graphicspath{{./Figures/}}

\setlength{\parindent}{0pt} 

%%%%%%%%%%%%%%%%%%%%%%%%%%% Discussion %%%%%%%%%%%%%%%%%%%%%%%%%%%%%%%%%%%%%%%%%%%%%

\section{Discussion}


\subsection{Gene expression signatures in CLL}
Due to the phenotypic and genotypic heterogeneity of CLL, gene expression profiling on CLL has yet provided little insight into molecular mechanisms underlying different genotypes.     
The PACE study represents the most comprehensive data set of CLL transcriptomes so far. We were able to distinguish gene expression profiles by the most prevalent genomic variants like IGHV hypermutations and Trisomy12 in an unsupervised approach. Groups classified by methylation pattern were clearly distinguishable by hierarchical clustering as well as PCA of gene expression data. These findings validate the quality of PACE and indicate its potential for detailed gene expression analysis. At the same time, they confirm distinct changes on gene expression level due to IGHV mutations and Trisomy12. We found 6.9\% of gene expression variability associated with IGHV status, in contrast to previous findings of only 1.5\% \citep{Ferreira2014}. This reveals the strength of our data set, due to the large sample size compared to previous studies.    
In total we analyzed gene expression profiles of the 13 most common single genetic variants and found distinct gene expression pattern for 8 of them. Thus, the PACE data set can serve as basis to study underlying molecular mechanisms in detail. 

\subsubsection{B cell receptor signaling is up regulated in IGHV-unmutated samples}
B cell receptor (BCR) signaling plays a decisive role in B cell survival and proliferation. CLL samples with unmutated IGHV genes showed increased activation and usually expressed polyreactive BCRs and high level of CD38 and Zap70, which are established clinical marker. As shown in previous studies IGHV mutated samples express more specific BCRs and show low activation of B cell receptor signaling \citep{Hacken2016}. Our Gene expression data confirmed these findings and we could show that up regulation of BCR signaling, CD38 and Zap70 in IGHV-unmutated samples proceeds already on transcriptome level. Downstream pathways like TNF alpha signaling via NFKB and PI3K-AKT-MTOR signaling were also enriched. Besides, we found the expression of tumor supressor genes, including TP63, TEAD1 and DIRAS3, increased in IGHV unmutated samples. This is in line with a prior finding that BCR signaling down regulates TP63 in CLL and therefore promotes tumor progression \citep{Humphries2013}. This regulation by BCR signaling is less prevalent in IGHV-mutated samples, where we found tumor suppressors down regulated on gene expression level.  


\subsection{Trisomy12 samples show increased expression of integrins driving lymphnode homing}
Even though Trisomy12 is a frequent and drastic chromosomal aberration in CLL the molecular mechanisms driving its pathology are not known. Recent studies found that lymphocyte adhesion cascade resulting in increased lymphnode homing of malignant B cells was activated in Trisomy12 samples \citep{Riches2014,Ganghammer2015}. The CD49d, subunit of VLA-4 is a major driver of the lymphocyte adhesion cascade \citep{Strati2017}. Furthermore the relevance of stimulating factors from the lymph node environment has been shown in many cell culture studies \citep{Hacken2016}. Gene expression data confirmed the up regulation of CD49d, as well as surface integrins ITGAM, ITGB2, ITGB1, ITGB7 and intracellular integrin signaling molecules like RABP-1. These surface integrins facilitate cellular adhesion and promote transendothelial migration of circulating cells into the lymph node. Our results support prior hypothesis that increased lymphonode homing acts as a major mechanism for tumor progression in CLLs with Trisomy12 and further confirm the relevance of the lymph node environment in CLL.         


\subsection{Down regulation of TP53 is associated with Wnt-signaling}
Deletion and mutation of TP53 lead to similar gene expression patterns, which can be associated to DNA damage response and Wnt-signaling. Most samples have a complete loss of functional TP53, but some exhibit only one dysfunctional allele. Those differed often in their gene expression from samples with a complete loss of TP53 function. In contrast to most of the other variants, the mechanism of tumor progression underlying TP53 loss of function is already known and related to its function for DNA damage control. Consistent with this, differentially expressed genes associated with TP53 were enriched in DNA damage response pathway. We further found Wnt/beta-Catenin signaling up regulated by TP53 loss. Deprived Wnt suppression by TP53 has already been observed in other cancers like prostate and colorectal cancer \citep{Kim2011a}. Furthermore, Wnt activation has been associated with amplified Myc activation, which is a known proto-oncogene driving cell proliferation \citep{Rennoll2015}. Remarkably, we also found up regulation of Myc targets on gene expression level. Altogether, those are strong evidence for dysregulated Wnt signaling as consequence of complete functional loss of TP53. This could impact tumor progression in other cancer types in a similar way, but further investigation is needed. One possible way is to investigate the role of mir-34, which is associated with Wnt supression via TP53. But micro-RNA detection in poly-A selected RNA sequencing data is not reliable and thus further experimental examination is necessary.         


\subsection{Del11q22 samples cluster by IGHV status}
Del11q22 samples formed a distinct cluster, characterized by the down regulation of ATM, BIRC2 and other genes on chromosome 11. This cluster correlated with IGHV unmutated status, while IGHV-mutated samples showed different gene expression pattern. Co-occurrence analysis already revealed biases in IGHV status and Del11q22. Furthermore we found marker genes like SEPT10 that had been associated with IGHV status differentially expressed in Del11q22 samples before \citep{Benedetti2007}. Together these results suggest an interaction among those variants, which share similar gene expression regulations and underlying mechanisms. ATM mutation is less frequent than Del11q22 and has been proposed as a later event that is often driven by monoallelic loss of the other ATM gene \citep{Landau2015}. We could not distinguish between mono- or biallelic loss of ATM function on gene expression level and did not find differential expression of genes involved in DNA damage control. This indicates a primary mechanism of Del11q22 in tumor progression independent of ATM driven DNA damage control, with ATM mutation as a later event in tumor manifestation.    

\subsection{Interferon signaling and the chaperone UQCC complex are transcriptional regulated in SF3B1 mutated CLLs}
SF3B1 is a splicing factor and related to aberrant RNA and ribosomal procession. A prior study that investigated transcriptomic regulation by differential exon usage revealed similar pattern as in melanoma samples \citep{Reyes}. They suggest a specific role of UQCC and RPL31 in transcriptional regulation of CLL and melanoma. UQCC is a chaperone involved in cytochrome B biosynthesis, but a specific relation to tumorigenesis has not been discovered yet. Interestingly we found UQCC1 gene differentially up regulated in SF3B1 mutated samples. Furthermore, differentially expressed genes were enriched in interferon alpha response, which is in line with findings from differential exon usage. Altogether this supports a functional role on transcriptional regulation of UQCC and interferon signaling as important mechanisms in SF3B1 mutated CLL samples. Overall we found distinct gene expression pattern. The investigation of isoforms and splice pattern could unravel further effects of transcriptional regulation and complement our findings. Furthermore, expression analysis of UQCC in melanoma could help to reveal its significance for tumor progression, as the common pattern in exon usage suggest similar underlying mechanisms. 

\subsection{Cell cycle progression in G2 phase as a potentially altered pathway in BRAF mutated CLLs}
Gene expression in BRAF mutated CLLs was less distinct compared to other genetic variants. Nevertheless, genes related to G2M checkpoint were coordinately up regulated. BRAF mutations are common in various cancer types and usually activate MAPK signaling via the RAS/RAF, MEK/ERK pathway. A direct role of oncogenic activated MAPK signaling for G2 progression and cell cycle activation has already been shown in thyroid follicular neoplasms and our analysis indicates it as a potential mechanism of tumor promotion in CLL \citep{Knauf2006}.

\subsection{Genome de-methylation defines subgroups of CLL with distinct gene expression pattern}
The degree of de-methylation separates CLL samples into groups of high-, intermediate- and low- programmed genomes. We show that these groups are reflected on transcriptional level with distinct gene expression pattern. Differential de-methylation of a set of multifunctional transcription factors has been proposed as tumor promoting mechanism determining these groups. Remarkably, we find 3 out of 4 transcription factors from this set differentially expressed between methylation groups. These findings support the relevance of methylation groups as well as their pathogenic function by differentially methylated multifunctional transcription factors. 


\subsection{Tumor epistasis in Trisomy12 and IGHV results in distinct expression pattern in line with drug response phenotypes}
A recent study by \citet{Bulian2017} on 951 CLL samples identified IGHV status as the most reliable prognostic marker in Trisomy12 CLLs. Overall survival and time to first treatment differed significantly among those four group of samples: U-CLLs with Trisomy 12, M-CLLs with Trisomy12, U-CLLs without Trisomy12 and M-CLLs without Trisomy12. Our data and analysis provide support for this observation. Trisomy12 samples formed an extra cluster regardless of IGHV status. In a model considering a linear combination of the genotype effects, we found 602 genes with significant epistatic expression pattern in samples with co-occurrence of Trisomy12 and IGHV hypermutations. These genes formed clusters that reflect different ways how the variants interact e.g. alleviating or aggravating another. Using analysis of variance we could infer directions describing the hierarchy of interaction. These directions were consistent within gene cluster, which supports our hypothesis of different layers of interaction between the genetic variants. Particularly, the group of Trisomy12 and IGHV mutated samples showed strong up regulation of a set of genes, including CD38 and MAP3K8(TPL2). A drug response study of the PACE data set found an increased sensitivity of Trisomy12 samples towards MEK/ERK inhibitors \citep{Dietrich}, which was rescued in M-CLLs. TPL2 directly activates ERK1/2 downstream of the usual Ras/Raf cascade as an alternative pathway \citep{Rousseau2016}. Gene expression data suggest increased TPL2 level in M-CLL as dominant activator of ERK as a way how Trisomy12 M-CLLs resist MEK/ERK inhibitors.  

\subsection{A multivariate model controls for confounding by other variants and gives insight into gene regulation with another resolution.}
Differential expression analysis of the effect of each variant solely resulted in strong reduction of significantly associated genes. Many of these genes were in line with previous findings or expectations, thus confirming the relevance of our results. Interestingly the ATM gene was associated to Del11q22 even when we controlled for the effect of ATM mutations. Similar to results from the original model the effect on gene expression in ATM mutated samples differed from the effect of ATM deletion in Del11q22 samples. This indicates that these variants have independent underlying mechanisms even if they are related to changes in the same gene. Another interesting finding is the relation of BRAF mutations to RRM2. Previous studies found a beneficial effect of combinatorial targeting of mutant BRAF and RRM2 in melanoma and our results suggest a similar relation, which could have implications for therapeutic strategies \citep{Fatkhutdinov2016}. In general results from likelihood ratio test with a multivariate model can be used to study direct gene-variant association and differentially expressed genes in detail, while the original model with degree of T cell contamination, IGHV status and Trisomy12 as covariate give a broader overview on observed effects in mechanistic changes and related pathways.    



\subsection{Biases and limitations}
Our gene expression analysis revealed distinct pattern associated with frequent genetic variants. By exploratory data analysis, we ensured the suitability to define gene signatures within the PACE data set. Nevertheless we were not able to define distinct gene signatures for all investigated variants. In the case of MED12 and Gain8p24/Del8q12, low occurrence was a major limitation. Furthermore co-occurrence and functional hierarchy biased the analysis, even though we accounted for dominant variances as co-variables in differential expression analysis. The likelihood ratio test of the multivariate approach controlled for co-occurring variants, but especially for functional related variants this resulted only in a few differentially expressed genes. A model with all variants as covariate and Wald test did not result in more distinct expression pattern.
Del13q14 is the most frequent genomic aberration, but we still did not find distinct interpretable gene expression pattern. This reveals weaknesses of RNA sequencing, which provides in depth insight into mRNA levels, but does not account for any other type of regulation. Furthermore, mRNA level does not always correlate with protein translation level, which is regulated by complex mechanisms on various levels. Our study showed that the influence of these regulators vary widely, with some genotypes being stronger associated to transcriptomic quantities than others. Furthermore, our results are biased by poly-A selection during library preparation not accounting for micro RNAs and other regulating RNAs. Our findings complement current understanding of CLLs by supporting previous hypothesis as well as providing novel hypothesis on molecular mechanisms underlying the pathogenesis of CLL, which can guide the design of further experimental studies 

%%%%%%%%%%%%%%%%%%%%%%%%%%% Conclusion %%%%%%%%%%%%%%%%%%%%%%%%%%%%%%%%%%%%%%%%%%%%%

\section{Conclusion}

In this study we analyzed gene signatures of frequent genetic variants in CLL samples to identify underlying molecular mechanisms. These mechanisms can be used to understand tumor development, progression and manifestation in CLL and provide potential targets of personalized medicine.\\

Despite clear clinical markers, CLL is a heterogeneous disease with complex genotypes and a variety of phenotypes with different clinical outcome. Here we aimed to unravel molecular pathways associated with these different genotypes. As the genetic heterogeneity is reflected by transcriptome data our results were limited by cohort size and dimensions of data. The PACE dataset is the most comprehensive transcriptome data collection so far, including 184 CLL samples with a variety of clinical data and metadata. This enabled us to detect expression pattern with new resolution (e.g. we found IGHV status associated with $6.9\%$ gene expression variability vs. $1.5\%$ in previous studies \citep{Ferreira2014}).\\   

Here we show distinct gene expression pattern of IGHV status, methylation groups and Trisomy12 in unsupervised clustering. This emphasizes the relevance and potential of transcriptome data to study molecular pathways in CLL. 
Consistent with previous findings, we show the marker genes in BCR pathway, CD38 and ZAP70, are up-regulated at gene expression level in U-CLLs and confirm the activation of B cell receptor signaling by enrichment analysis. Methylation groups are reflected by distinct gene expression, especially in a set of highly variable genes. This supports the hypothesis that activated multi-functional transcription factors are responsible for the distinct phenotypes of methylation groups through regulating DNA de-methylation. We found EGR1, NFAT and EBF1 to be differentially expressed, supporting their function as transcriptional regulators. In Trisomy12 samples, we detected up regulation of surface integrins and adhesion molecules, proposing increased cell migration. The importance of environmental stimulation in CLL progression has been shown before. Our results indicate enhanced lymph node homing as pathogenic pathway in Trisomy12 samples.  
Besides, we found evidence for Wnt-suppression-dependent-Myc-activation as a tumor promoting mechanism in CLLs with functional loss of TP53. Even in less frequent variants such as BRAF mutations, we were able to detect aberrant cell cycle progression via G2M checkpoint regulation as differentially expressed molecular mechanisms and hypothesized a function in tumor progression. In line with a prior study of differential exon usage in SF3B1 mutated CLL samples, we found interferon signaling and the chaperone UQCC differentially expressed, suggesting functionally different transcriptional regulation. To identify genes associated to each variant solely we tested a multivariate model with all variants against a reduced model with confounding factors in a likelihood ratio test. This enabled us to detect changes in single genes and study direct gene-variant association and differentially expressed genes in detail. Finally, we used a tumor epistasis model to investigate interactivity between different genetic variants on gene expression level. This revealed distinct pattern between IGHV status and Trisomy12. We identified gene clusters representing different effects of epistatic interaction and were able to confirm them by inferring directions using variance analysis. On this basis, we hypothesize an alternative activation of ERK1/2 via TPL2 as mechanism behind MEK/ERK inhibitor resistance in trisomy12 M-CLL samples.\\

In conclusion, we proved the relevance of gene expression regulation in the pathogenesis of CLL. By detailed analysis, we were able to detect gene expression pattern that can be related to genotype data and clinical features, revealing the value of such studies. We show that even interactivity of genetic variants is reflected and can be studied on gene expression level. Our findings provide valuable insight on the molecular mechanism underlying CLL pathogenesis and can guide the design of further experimental studies.
